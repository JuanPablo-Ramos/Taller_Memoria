\documentclass{article}
\usepackage[utf8]{inputenc}
\usepackage[spanish]{babel}
\usepackage{listings}
\usepackage{graphicx}
\graphicspath{ {images/} }
\usepackage{cite}

\begin{document}

\begin{titlepage}
    \begin{center}
        \vspace*{1cm}
            
        \Huge
        \textbf{TALLER - Nociones De La Memoria Del Computador}
            
        \vspace{1cm}
        \LARGE
        Informatica ll
            
        \vspace{1.5cm}
            
        \textbf{Juan Pablo Ramos Vélez}
            
        \vfill
        \begin{figure}[h]
        \includegraphics[width=4cm]{UdeA logo.png}
        \centering
        \label{fig:UdeA logo}
        \end{figure}  
            
        \vspace{1.5cm}
            
        \Large
        Despartamento de Ingeniería Electrónica y Telecomunicaciones\\
        Universidad de Antioquia\\
        Medellín\\
        Septiembre de 2020
            
    \end{center}
\end{titlepage}

\begin{center}
    \bf{\sc\Huge Memoria }\\
\end{center}


\vspace{1cm}

\large
La memoria del computador es el espacio de almacenamiento donde se procesan los datos y se almacenan las instrucciones necesarias para el procesamiento, el propósito del almacenamiento es guardar los datos que el computador no esté usando, la memoria tiene muchas ventajas, por ejemplo, al apagarse el computador, la información se retiene. 

\vspace{0.5cm}

Hay muchos tipos de memorias, la memoria ROM, la memoria RAM y las memorias externas 
La memoria RAM (Random Acces Memory- Memoria de acceso aleatorio) es donde el ordenador guarda los datos que está utilizando en ese momento, el procesador accede a un lugar aleatorio de la memoria sin tener que pasar por la información anterior o posterior. Esta memoria se actualiza mientras el computador está en uso, cuando el computador se apaga, esta pierde sus datos.

\vspace{0.5cm}

La memoria ROM (Read Only Memory- Memoria de solo lectura), es donde se puede almacenar la información necesaria para el inicio del proceso de arranque, este tipo de memoria puede conservar los datos que tiene aun cuando el computador este apagado.
Las memorias externas o auxiliares son todos los dispositivos y medios de almacenamiento que no son parte de la memoria interna del computador, la memoria externa más utilizada es el disco duro.

\vspace{0.5cm}

La memoria es uno de los principales recursos de la computadora, actualmente la mayoría de los sistemas cuentan con una gran capacidad de memoria, pero hay algunas aplicaciones que tienen alto requerimiento, lo que genera unos escases de memoria en los sistemas multitarea, el administrador de memoria se encarga de asignar memoria cuando se necesite y cuando no, de liberarla.

\vspace{2.5cm}
Cuando pensamos en la capacidad de almacenamiento de un computador, se piensa en el procesador o en la memoria RAM, por eso, cuando se quiere aumentar la velocidad de un computador, lo primero que se hace es aumentar la RAM, pero lo que más influye en la velocidad es el disco duro, la razón está en la mecánica de funcionamiento.
Los datos del disco se graban y se leen mediante una cabeza de lectura, esta es la que hace que se tarde más o menos tiempo, ya que la velocidad depende en que tan rápido se tarde el cabezal en dirigirse al sector en el que los datos están almacenados.
Otro factor importante en la velocidad de la memoria es la velocidad del bus. Este es el circuito que conduce la información de un dispositivo a otro sobre la placa madre, especialmente el llamado reloj, que conecta a la memoria con el microprocesador. Dependiendo de la cantidad de bits que pueda transmitir el bus, será mayor o menor la velocidad de la memoria.

\vspace{2cm}

\cite{Vasquez}
\bibliographystyle{IEEEtran}
\bibliography{references}
http://www.aliat.org.mx/BibliotecasDigitales/sistemas/Arquitectura_computadoras_I.pdf
\end{document}
